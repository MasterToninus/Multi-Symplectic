\title{(An introduction to )MultiSymplectic Geometry and Classical Field systems}
\author{
	International Ph.D. Program in "Science" \\
	"Differential geometry and applications to modern physics" \\
	Department of Mathematics and Physics\\
	Universit\'a Cattolica del Sacro Cuore\\
	via Musei 41, 25121 Brescia, \underline{Italy}
}

\documentclass[a4paper,12pt,fleqn]{article}  %Per La Stampa
\usepackage[a4paper, margin=3cm]{geometry}

\usepackage{amsmath}
\usepackage[multisym, geomec, basic, diffgeo]{./Math-Symbols-List/toninus-math-symbols}
\usepackage[subpreambles=true]{standalone}
\usepackage{commath}

\usepackage[italian,english]{babel}
\usepackage[utf8]{inputenc}

\usepackage{graphicx}
\usepackage{hyperref}

%\usepackage{Latex-Theorem/theoremtemplate}

%	\renewcommand{\AffDualJet}{ J^{1 \TwistedAffine}E }
%	\renewcommand{\LinDualJet}{ \overrightarrow{J^{1 \TwistedLinear}E} }

\providecommand{\oast}{\ensuremath{\:\!%
 {\raisebox{0.1ex}{$\scriptscriptstyle \bigcirc$} \hspace{-0.5em} \ast}}}
\providecommand{\ostar}{\ensuremath{\:\!%
 {\raisebox{0.05ex}{$\scriptscriptstyle \bigcirc$} \hspace{-0.5em} \star}}}
\providecommand{\smoast}{\ensuremath{%
 {\raisebox{-0.25ex}{$\textstyle \circ$} \hspace{-0.42em}
 {\scriptscriptstyle \ast}}}}
\providecommand{\smostar}{\ensuremath{%
 {\raisebox{-0.25ex}{$\textstyle \circ$} \hspace{-0.42em}
  \raisebox{0.05ex}{$\scriptscriptstyle \star$}}}}

\begin{document}
\maketitle

$$
\ast 
\star 
\oast 
\ostar
\smoast
\smostar
J^\smostar
J^\smoast
J^\ostar
$$


\begin{abstract}
$n$-plectic structures (also called \emph{multisymplectic}) are a rather straightforward generalization of symplectic ones where closed non-degenerate $n+1$-forms take the place of  $2$-forms.
\\
As the same way as one can relate a symplectic manifold to any ordinary classical mechanical system (e.g. a single point-like particle constrained to some manifold), it is possible to associate a multisymplectic manifold to any classical field systems (think them as a continuum medium like a filament or a membrane).
\\
The aim of my talk is to give an account on the multisymplectic framework for (I-order) classical fields theories comparing it with another object that plays a significant role in the mathematical description of classical fields called \emph{Covariant phase space}.
\\
Being the latter a sort of "$\infty$-dimensional manifold" (namely a mapping space), we will draw from this picture the idea that multisymplectic geometry could be seen as a tool that allow us to treat such formal object in a finite dimensional setting.


\end{abstract}

%------------------------------------------------------------------------------------------------
% Bibliography (BibTex)
% https://arxiv.org/hypertex/bibstyles/
%------------------------------------------------------------------------------------------------
			\nocite{*}
			\bibliographystyle{ieeetr}
			\bibliography{biblio}
%------------------------------------------------------------------------------------------------

\section{A glance at the geometric approach to classical mechanics}
Recall: Geometric Mechanics of Classical system
Configuration space
Generalized velocities space and phase space
...



\section{Far capire cosa intendo per campo classico}
Classicamente: DOF come numero di misure necessarie per fissare univocamente una configurazione.
\begin{table}[]
\begin{tabular}{lll}
	System & Configuration space (locally) & protoype 
 	\\
	Ordinary system & $\mathbb{R}^n$ with $n\in \mathbb{N} = Hom_{set}(\{1,\ldots n\}, \mathbb{R})$ & A point on a $n$-dimensional mfd.
	\\
	1 dimensional field & $\mathbb{R}^\mathbb{R} = C^\infty(\mathbb{R},\mathbb{R})$ & collection of points on a line each one parametrized by ...
	\\
	n dimensional field & $(\mathbb{R}^m)^{\mathbb{R}^n} = C^\infty (\mathbb{R}^n, \mathbb{R}^m)$ & ...
\end{tabular}
\end{table}

A questo punto si dovrebbe Essere chiaro il punto di partenza:
\\

\section{Kinematics of Classical fields}
The starting point is the \emph{configuration bundle} 
 	\begin{displaymath}
 		E = \left( \pi_{M,E} : E\rightarrow M \right)
 	\end{displaymath}
is a smooth bundle encoding the kinematics.

\subsection{Jet Bundle}
The first jet bundle of $E$ generalises the space of generalized velocities.

The fiber over a point $y\in E$ is
\begin{displaymath}
	\left( J^1 E\right)_y = \left\lbrace \gamma \in \text{L}(T_{\pi(y)}M, T_y E) \; \vert \: T\pi \circ \gamma = \text{id}_{T_{\pi(x)} M} \right\rbrace
\end{displaymath}
(T = Tangent functor L = space of linear maps)

which is and affine subspace of $\text{L}(T_{\pi(y)}M, T_y E)$ modelled on the following (difference) vector space:
\begin{displaymath}
	\vec{\left( J^1 E\right)_y} = \left\lbrace z \in \text{L}(T_{\pi(y)}M, T_y E) \; \vert \: T\pi \circ z = 0 \right\rbrace
	\cong text{L}(T_{\pi(y)}M, T_y E)
\end{displaymath}

\section{Spunti / Open problems}
\begin{itemize}
	\item We have a momap on ms mfd and a momap on mapping space (as diffeological space) does the two match?
	\item We have a canonical m-s structure on J.. and a canonical poisson structure con cov phase space (Peierls bracket) does the two match?
	\item there are other notion of moment map on the multisymplectic setting. For instance how the definition in section 4 of \cite{Gotay1998a} is related to the one in \cite{Ryvkin2018}?
\end{itemize}
	
	

\section{Spunti dei Saggi}
\subsection{Rogers: arXiv:1009.2975v3}
Multisymplectic manifolds naturally arise in certain covariant Hamiltonian formalisms for classical field theory.
In these formalisms, one describes a $(n+1)$-dimensional field theory by using a finite-dimensional $n$-plectic manifold as a \emph{multi-phase space} instead of an infinite-dimensional phase space. 
The $n$-plectic form can be used to define a system of partial differential equations which are the analogue of Hamilton's equations in classical mechanics. 
The solutions to these
equations correspond to particular submanifolds of the multi-phase space that encode the value of the field at each point in space-time as well as the values of its time and spatial derivatives.

\subsection{Forger-Romero}
The two formalisms, although both fully covariant and directed towards
the same ulti\-mate goal, are of different nature; each of them has its
own merits and drawbacks.
\begin{itemize}
 \item The multisymplectic formalism is manifestly consistent with the
       basic principles of field theory, preserving full covariance, and
       it is mathematically rigorous because it uses well established
       methods from calculus on finite-dimensional manifolds. On~the
       other hand, it does not seem to permit any obvious definition
       of the \mbox{Poisson} bracket between observables. Even the
       question of what mathematical objects should represent physical
       observables is not totally clear and has in fact been the subject
       of much debate in the literature. Moreover, the introduction of
       $n$ conjugate momenta for each coordinate obscures the usual
       duality between canonically conjugate variables (such as momenta
       and positions), which plays a fundamental role in all known methods
       of quantization. A definite solution to these problems has yet to be
       found.
 \item The covariant functional formalism fits neatly into the philosophy
       underlying the symplectic formalism in general; in particular, it
       admits a natural definition of the Poisson bracket (due to Peierls
       \cite{Pe} and further elaborated by DeWitt \cite{DW1,DW2,DW3})
       \linebreak that preserves the duality between canonically conjugate
       variables. Its main drawback is the lack of mathematical rigor, since
       it is often restricted to the formal extrapolation of techniques from
       ordinary calculus on manifolds to the infinite-dimensional setting:
       transforming such formal results into mathematical theorems is a
       separate problem, often highly complex and difficult.
\end{itemize}

Multiphase space (ordinary as well as extended) is the geometric environment
built by appropriately patching together local coordinate systems of the form
$(q^i,p\>\!_i^\mu)$~-- in- \linebreak stead of the canonically conjugate
variables $(q^i,p_i)$ of mechanics~-- together with space-time coordinates
$x^\mu$ and, in the extended version, a further energy type variable that
we shall denote by $p$ (without any index). The global construction of
these multiphase spaces, however, has only gradually come to light; it
is based on the following mathematical concepts.
\begin{itemize}
 \item The collection of all fields in a given theory, defined over a fixed
       ($n$-dimensional orientable) space-time manifold $M$, is represented
       by the sections $\varphi$ of a given fiber bundle $E$ over $M$, with
       bundle projection $\, \pi : E \rightarrow M \,$ and typical fiber $Q$.
       This bundle will be referred to as the \emph{configuration bundle}
       of the theory since $Q$ corresponds to the configuration space of
       possible field values.
 \item The collection of all fields together with their partial derivatives
       up to a certain order, say order $r$, is represented by the $r$-jets
       $\, j^{\,r\!} \varphi \equiv (\varphi,\partial\varphi,\ldots,
       \partial^{\,r\!} \varphi)$  of sections of $E$, which are
       themselves sections of the $r^{\mathrm{th}}$ order jet bundle
       $J^r E$ of $E$, regarded as a fiber bundle over~$M$. In this paper,
       we shall only need first order jet bundles, \linebreak with one
       notable exception: the global formulation of the Euler\,-\,Lagrange
       equations requires introducing the second order jet bundle.
 \item Dualization -- the concept needed to pass from the Lagrangian to
       the Hamiltonian framework via the Legendre transformation -- comes
       in two variants, based on the fundamental observation that the first
       order jet bundle $J^1 E$ of $E$ is an affine bundle over $E$ whose
       difference vector bundle $\vec{J}^{\,1} E$ will be referred to as
       the linear jet bundle. Ordinary multiphase space is obtained as
       the twisted linear dual $\vec{J}^{\,1\oast} E$ of $\vec{J}^{\,1} E$
       while extended multiphase space is obtained as the twisted affine
       dual $J^{1\ostar} E$ of $J^1 E$, where the prefix ``twisted'' refers
       to the necessity of taking an additional tensor product with the
       bundle of $n$-forms on $M$.\footnote{We use an asterisk $\ast$ to
       denote linear duals of vector spaces or bundles and a star $\star$
       to denote affine duals of affine spaces or bundles. These symbols
       are appropriately encircled to characterize twisted duals, as
       opposed to the ordinary duals defined in terms of linear or
       affine maps with values in $\mathbb{R}$.}
 \item The Lagrangian $\mathcal{L}$ is a function on $J^1 E$ with values
       in the bundle of $n$-forms on~$M$ so that it may be integrated to
       provide an action functional which enters the variational principle.
       The De Donder\,-\,Weyl Hamiltonian $\mathcal{H}$ is a section of
       $J^{1\ostar} E$, considered as an affine line bundle over
       $\vec{J}^{\,1\oast} E$.
\end{itemize}
Note that the formalism is set up so as to require no additional structure
on the con\-figuration bundle or on any other bundle constructed from it:
all are merely fiber bundles over the space-time manifold $M$. Of course,
additional structures do arise when one is dealing with special classes
of fields (matter fields and the metric tensor in general relativity are
sections of vector bundles, connections are sections of affine bundles,
nonlinear fields such as those arising in the sigma model are sections
of trivial fiber bundles with a fixed Riemannian metric on the fibers,
etc.), but such additional structures depend on the kind of theory
considered and thus are not universal. Finally, the restriction
imposed on the order of the jet bundles considered reflects the
fact that almost all known examples of field theories are governed by
second order partial differential equations which can be derived from a
Lagrangian that depends only on the fields and their partial derivatives
of first order, which is why it is reasonable to develop the general
theory on the basis of a first order formalism, as is done in mechanics
\cite{AM,Arn}.

A fundamental prerequisite for setting up a general mathematical framework
for geo\-metric field theory is to obtain a clear picture of what should be
the most general concept of a (classical) field. In physics, we distinguish
basically four different types of fields.
\begin{itemize}
 \item \emph{Matter fields} are fields that, in the context of quantum theory
       and in particular of the standard model, are associated with particles
       that constitute matter -- quarks and leptons.
 \item \emph{Gauge fields} are fields that, in the context of quantum theory
       and in particular of the standard model, are associated with particles
       that mediate three of the four fundamental interactions: the photon
       $\gamma$ for the electromagnetic interaction, the three intermediate
       vector bosons $W^\pm$ and $Z^0$ for the weak interaction and the
       eight gluons for the strong interaction. Among the three, only
       one -- the electromagnetic interaction -- describes a long-range
       force and therefore plays an important role also in the macroscopic
       world, where it is governed by Maxwell's theory of electromagnetism.
 \item \emph{Gravitational fields} occupy a particular position because
       according to general relativity as formulated by Einstein, they
       are related to the metric properties of space-time itself.
       Historically, general relativity has been the first example
       of a geometric field theory.
 \item \emph{Nonlinear fields} are fields taking values in some nonlinear
       manifold, often obtained as a submanifold of some vector space.
       Typical examples are the Heisenberg model (the continuum limit
       of the Heisenberg spin chain) and the sigma model.
\end{itemize}
Obviously, a general mathematical formalism for classical field theory
should be based on concepts capable of accomodating all these types of
fields. This leads to the definition of a field as being a section of
a (general) fiber bundle~$E$ over space-time~$M$, without additional
structure, since different types of fields correspond to sections of
different types of fiber bundles.
\begin{itemize}
 \item \emph{Matter fields} are sections of some given vector bundle over $M$,
       built from the tangent bundle $TM$ of $M$ together with its descendants
       (to accomodate the tensorial or spinorial character of these fields)
       and possibly from vector bundles $V$ over $M$ associated to some
       principal bundle $P$ over $M$ whose structure group describes
       internal symmetries.
 \item \emph{Gauge fields} are usually defined as connection forms on the
       total space of some given principal bundle $P$ over $M$ but can be
       reinterpreted as sections of the connection bundle of $P$ -- an
       affine bundle $CP$ over $M$ constructed canonically from $P$.
       See for example \cite{NODG}.
 \item \emph{Gravitational fields} are traditionally described by a metric
       tensor which, again, is a section of a vector bundle over $M$, namely
       the symmetric second tensor power of the cotangent bundle $T^* M$ of
       $M$, but subject to the condition of non-degeneracy and Lorentzian
       signature which restricts it to take values in a certain open
       subbundle of the latter.
 \item \emph{Nonlinear fields} are usually sections of a globally trivial
       bundle $\, E = M \times Q \,$ where $Q$ is a given manifold; such
       sections can be identified with maps from $M$ to $Q$. In the most
       general version of the sigma model, $Q$ is just a Riemannian manifold.
\end{itemize}


\nocite{*}
\bibliographystyle{alpha}
\bibliography{biblio}



\end{document}
