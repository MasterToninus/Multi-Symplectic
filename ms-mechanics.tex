\title{(An introduction to )MultiSymplectic Geometry and Classical Field systems}
\author{
	International Ph.D. Program in "Science" \\
	"Differential geometry and applications to modern physics" \\
	%
	\begin{minipage}[c]{0.5\textwidth}
		\vspace{2em}
		\centering
		Department of Mathematics and Physics\\
		Universit\'a Cattolica del Sacro Cuore\\
		via Musei 41, 25121 Brescia, \underline{Italy}
	\end{minipage}
	\hfill
	\begin{minipage}[c]{0.5\textwidth}
		\vspace{2em}
		\centering
		Department of Mathematics\\
		KU Leuven\\
		Celestijnenlaan 200B, 3001 Leuven, \underline{Belgium}
	\end{minipage}
}

\documentclass[a4paper,12pt,fleqn]{article}  %Per La Stampa
\usepackage[a4paper, margin=3cm]{geometry}

\usepackage{amsmath}
\usepackage{amsfonts}
%\usepackage[multisym, geomec, basic, diffgeo]{./Math-Symbols-List/toninus-math-symbols}
\usepackage[subpreambles=true]{standalone}
\usepackage{commath}

\usepackage[italian,english]{babel}
\usepackage[utf8]{inputenc}

\usepackage{graphicx}
\usepackage{hyperref}
\usepackage{tabularx}

\providecommand{\oast}{\ensuremath{\:\!%
 {\raisebox{0.1ex}{$\scriptscriptstyle \bigcirc$} \hspace{-0.5em} \ast}}}
\providecommand{\ostar}{\ensuremath{\:\!%
 {\raisebox{0.05ex}{$\scriptscriptstyle \bigcirc$} \hspace{-0.5em} \star}}}
\providecommand{\smoast}{\ensuremath{%
 {\raisebox{-0.25ex}{$\textstyle \circ$} \hspace{-0.42em}
 {\scriptscriptstyle \ast}}}}
\providecommand{\smostar}{\ensuremath{%
 {\raisebox{-0.25ex}{$\textstyle \circ$} \hspace{-0.42em}
  \raisebox{0.05ex}{$\scriptscriptstyle \star$}}}}

\begin{document}
\maketitle

$$
\ast 
\star 
\oast 
\ostar
\smoast
\smostar
J^\smostar
J^\smoast
%J^\ostar
$$


\begin{abstract}
$n$-plectic structures (also called \emph{multisymplectic}) are a rather straightforward generalization of symplectic ones where closed non-degenerate $n+1$-forms take the place of  $2$-forms.
\\
Just as one can associate a symplectic manifold to an ordinary classical mechanical system (e.g. a single point-like particle constrained to some manifold), it is possible to associate a multisymplectic manifold to any classical field systems (e.g. a continuous medium like a filament or a membrane).
\\
The aim of this talk is to give an account on the multisymplectic framework for (I-order) classical fields theories trying to comparing it with another object that plays a significant role in the mathematical description of classical fields called \emph{Covariant phase space}.
\\
Being the latter a sort of "$\infty$-dimensional manifold" (namely a mapping space), we will draw from this picture the idea that multisymplectic geometry could be seen as a tool that allows us to treat such formal object in a finite-dimensional setting.


\end{abstract}


\section{Preliminiaries}

%What's geometric Mechanics
\subsection{A glance at the geometric approach to classical mechanics}
The context in which we are going to work is \emph{Geometric Mechanics}.
\\
Roughly speaking, geometric mechanics is a branch of (applied) mathematics that employs differential geometry to the description of physical systems. Unlike analytical mechanics, the focus is on encoding the mechanical properties of a physical system regardless of the reference frame (coordinate system).
\\
Let us briefly recall what are the most important ingredients of an ordinary classical\footnote{Here "ordinary" stands for finite degrees of freedom and "classical" stands for non-quantum and non-relativistic.} system.

\begin{center}
  %\includestandalone{Pictures/Figure_ordinary_landscape}
\end{center}

\begin{itemize}
	\item The starting point is the \emph{Configuration space}, that is a smooth manifolds $Q$ containing all possible spatial displacements of a system.\\
	E.g. for the simple planar pendulum $Q= S^1$, for a double pendulum $Q =\mathbb{T}^2$
	\item A point in $Q$ is a statical configuration, that is opposed to the elements of the \emph{space of kinematics configuration}
	$$ \mathcal{C} = C^\infty(\mathbb{R},Q)$$
	which is the collection of all trajectories admitted by the constraints.
	\item The tangent bundle $TQ$ takes the name of \emph{space of generalized velocities}
	\item The cotangent bundle $T^\ast Q$ is called \emph{phase space}. Such space carries a natural symplectic form $\omega$ (\emph{Poincare\'} 2-form) defined in term of the tautological 1-form $$\omega= \text{d} \theta$$
\end{itemize}
%
All of this encode the geometry of the kinematics. Fixing a point in the previous two spaces is tantamount to give a complete set of initial data determining unambiguously the evolution of the system. 
\\
Such point carries more information that the simple spatial displacement and are interpreted as the \emph{physical state} of the system.
%
\begin{itemize}
\item Dynamics is encoded by two smooth functions:
	\\ 
	$L:TQ\rightarrow \mathbb{R}$ in the \emph{Lagrangian} framework
	\\
	$H:T^\ast Q \rightarrow \mathbb{R}$ in the \emph{Hamiltonian} framework
	\\
	Generally the two framework are not equivalent.
\item From the Lagrangian $L$ or the Hamiltonian $H$ one can obtain the so-called \emph{Action functional} $S:\mathcal{C}\rightarrow \mathbb{R}$ which yields the equation of motion through the d'Alambert principle.
\end{itemize}
%
The goal of this talk is to describe the analogue of this framework in the case of classical field systems.

%What's a Classical Field
\subsection{Modelling a classical field theory}
A classical field theory is a physical theory that predicts how "physical fields" evolves according to their interaction with the matter and themselves.
\\
The term "classical" is commonly reserved to theories treating electro-magnetism and gravitation. Here it is meant as non-quantum mechanics of physical fields or continuous systems.

Roughly, a physical field can be thought as the assignment of a physical quantity at each point of space and time.


Classicamente: DOF come numero di misure necessarie per fissare univocamente una configurazione.

\begin{tabularx}{\textwidth}{c X c c}
	System & Prototype & DoF & Configuration space (locally)\\
	Ordinary classical mech system & point-particle on a  $q$-dim manifold &
	$q$ finite (discrete) & $\mathbb{R}^n = Hom_{set}(\{1,\ldots n\}, \mathbb{R})$ \\
	1-dim. field & collection of points on a line each one parametrized by a point on the manifold $Q$ & $q$ for each point on the continuous line (infinite and continuous) &
	...
%	\\
%	n dimensional field & $(\mathbb{R}^m)^{\mathbb{R}^n} = C^\infty (\mathbb{R}^n, \mathbb{R}^m)$ & ...
\end{tabularx}


A questo punto si dovrebbe Essere chiaro il punto di partenza:
\\

\subsection{Multisymplectic geometry}


\section{Multymplectic approach to classical field theories}
Tabella -Dictionary


\subsection{Configuration Bundle}
-Kinematics of Classical fields-
The starting point is the \emph{configuration bundle} 
 	\begin{displaymath}
 		E = \left( \pi_{M,E} : E\rightarrow M \right)
 	\end{displaymath}
is a smooth bundle encoding the kinematics.

\subsection{Jet Bundle}
The first jet bundle of $E$ generalises the space of generalized velocities.

The fiber over a point $y\in E$ is
\begin{displaymath}
	\left( J^1 E\right)_y = \left\lbrace \gamma \in \text{L}(T_{\pi(y)}M, T_y E) \; \vert \: T\pi \circ \gamma = \text{id}_{T_{\pi(x)} M} \right\rbrace
\end{displaymath}
(T = Tangent functor L = space of linear maps)

which is and affine subspace of $\text{L}(T_{\pi(y)}M, T_y E)$ modelled on the following (difference) vector space:
\begin{displaymath}
	\vec{\left( J^1 E\right)_y} = \left\lbrace z \in \text{L}(T_{\pi(y)}M, T_y E) \; \vert \: T\pi \circ z = 0 \right\rbrace
	\cong text{L}(T_{\pi(y)}M, T_y E)
\end{displaymath}

\subsection{Dualization of Jet bundles}

\subsection{Lagrangian Dynamics}

\subsection{Hamiltonian Dynamics}

\subsection{Legendre transformation}

\subsection{examples}

\section{Spunti}

\subsection{Link to the Covariant Approach}

\subsection{Open problems?}
\begin{itemize}
	\item We have a momap on ms mfd and a momap on mapping space (as diffeological space) does the two match?
	\item We have a canonical m-s structure on J.. and a canonical poisson structure con cov phase space (Peierls bracket) does the two match? \\
		Yes! this is actually the result of \cite{forgeromero}. At pages 396-397 they show how to transgress the multisymplectic form to a symplectic form on Sol.\\
		In theorem 4 pag 403 they show that this form coincide with the one by Peierls constructed a' la Dewitt.

	\item there are other notion of moment map on the multisymplectic setting. For instance how the definition in section 4 of \cite{Gotay1998a} is related to the one in \cite{Ryvkin2018}?

Marco ha fatto un'osservazione interessante: la densità hamiltoniana come definita da Forger-Romero non sembra a priori collegata a nessun elemento delle "L-infty algebra" usata nei suoi lavori. Questo e' in contrasto con il parallelismo che si vorrebbe fare tra algebra di poisson per i sistemi tipo di particella e L-infity algebra per i sistemi tipo campo.

Anche Forger e Romero indicano come che uno dei grandi difetti del formalismo multi-simplettico il fatto che non sia chiaro quale sia la corretta struttura algebrica degli osservabili. Questo problema non è di poco conto visto che la maggior parte degli schemi di quantizzazione partono dal quantizzare l'algebra degli osservabili.

Marco ha parlato con Gotay: dice che la forma ms per i campi è molto particolare (esatta e definita su un particolare fibrato) e che lo studio della geo multisymp a livello molto generale potrebbe alla fine non essere centrale o utile per quanto riguarda la meccanica.	
	

\end{itemize}


%------------------------------------------------------------------------------------------------
% Bibliography (BibTex)
% https://arxiv.org/hypertex/bibstyles/
%------------------------------------------------------------------------------------------------
			\nocite{*}
			\bibliographystyle{ieeetr}
			\bibliography{biblio}
%------------------------------------------------------------------------------------------------




\end{document}
