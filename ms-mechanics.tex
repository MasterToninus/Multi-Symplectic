\title{(An introduction to )MultiSymplectic Geometry and Classical Field systems}
\author{
	International Ph.D. Program in "Science" \\
	"Differential geometry and applications to modern physics" \\
	%
	\begin{minipage}[c]{0.5\textwidth}
		\vspace{2em}
		\centering
		Department of Mathematics and Physics\\
		Universit\'a Cattolica del Sacro Cuore\\
		via Musei 41, 25121 Brescia, \underline{Italy}
	\end{minipage}
	\hfill
	\begin{minipage}[c]{0.5\textwidth}
		\vspace{2em}
		\centering
		Department of Mathematics\\
		KU Leuven\\
		Celestijnenlaan 200B, 3001 Leuven, \underline{Belgium}
	\end{minipage}
}

\documentclass[a4paper,12pt,fleqn]{article}  %Per La Stampa
\usepackage[a4paper, margin=3cm]{geometry}

\usepackage{amsmath}
\usepackage[multisym, geomec, basic, diffgeo]{./Math-Symbols-List/toninus-math-symbols}
\usepackage[subpreambles=true]{standalone}
\usepackage{commath}

\usepackage[italian,english]{babel}
\usepackage[utf8]{inputenc}

\usepackage{graphicx}
\usepackage{hyperref}

%\usepackage{Latex-Theorem/theoremtemplate}

%	\renewcommand{\AffDualJet}{ J^{1 \TwistedAffine}E }
%	\renewcommand{\LinDualJet}{ \overrightarrow{J^{1 \TwistedLinear}E} }

\providecommand{\oast}{\ensuremath{\:\!%
 {\raisebox{0.1ex}{$\scriptscriptstyle \bigcirc$} \hspace{-0.5em} \ast}}}
\providecommand{\ostar}{\ensuremath{\:\!%
 {\raisebox{0.05ex}{$\scriptscriptstyle \bigcirc$} \hspace{-0.5em} \star}}}
\providecommand{\smoast}{\ensuremath{%
 {\raisebox{-0.25ex}{$\textstyle \circ$} \hspace{-0.42em}
 {\scriptscriptstyle \ast}}}}
\providecommand{\smostar}{\ensuremath{%
 {\raisebox{-0.25ex}{$\textstyle \circ$} \hspace{-0.42em}
  \raisebox{0.05ex}{$\scriptscriptstyle \star$}}}}

\begin{document}
\maketitle

$$
\ast 
\star 
\oast 
\ostar
\smoast
\smostar
J^\smostar
J^\smoast
J^\ostar
$$


\begin{abstract}
$n$-plectic structures (also called \emph{multisymplectic}) are a rather straightforward generalization of symplectic ones where closed non-degenerate $n+1$-forms take the place of  $2$-forms.
\\
Just as one can associate a symplectic manifold to an ordinary classical mechanical system (e.g. a single point-like particle constrained to some manifold), it is possible to associate a multisymplectic manifold to any classical field systems (e.g. a continuous medium like a filament or a membrane).
\\
The aim of this talk is to give an account on the multisymplectic framework for (I-order) classical fields theories trying to comparing it with another object that plays a significant role in the mathematical description of classical fields called \emph{Covariant phase space}.
\\
Being the latter a sort of "$\infty$-dimensional manifold" (namely a mapping space), we will draw from this picture the idea that multisymplectic geometry could be seen as a tool that allows us to treat such formal object in a finite-dimensional setting.


\end{abstract}

%------------------------------------------------------------------------------------------------
% Bibliography (BibTex)
% https://arxiv.org/hypertex/bibstyles/
%------------------------------------------------------------------------------------------------
			\nocite{*}
			\bibliographystyle{ieeetr}
			\bibliography{biblio}
%------------------------------------------------------------------------------------------------

\section{A glance at the geometric approach to classical mechanics}
Recall: Geometric Mechanics of Classical system
Configuration space
Generalized velocities space and phase space
...



\section{Far capire cosa intendo per campo classico}
Classicamente: DOF come numero di misure necessarie per fissare univocamente una configurazione.
\begin{table}[]
\begin{tabular}{lll}
	System & Configuration space (locally) & protoype 
 	\\
	Ordinary system & $\mathbb{R}^n$ with $n\in \mathbb{N} = Hom_{set}(\{1,\ldots n\}, \mathbb{R})$ & A point on a $n$-dimensional mfd.
	\\
	1 dimensional field & $\mathbb{R}^\mathbb{R} = C^\infty(\mathbb{R},\mathbb{R})$ & collection of points on a line each one parametrized by ...
	\\
	n dimensional field & $(\mathbb{R}^m)^{\mathbb{R}^n} = C^\infty (\mathbb{R}^n, \mathbb{R}^m)$ & ...
\end{tabular}
\end{table}

A questo punto si dovrebbe Essere chiaro il punto di partenza:
\\

\section{Kinematics of Classical fields}
The starting point is the \emph{configuration bundle} 
 	\begin{displaymath}
 		E = \left( \pi_{M,E} : E\rightarrow M \right)
 	\end{displaymath}
is a smooth bundle encoding the kinematics.

\subsection{Jet Bundle}
The first jet bundle of $E$ generalises the space of generalized velocities.

The fiber over a point $y\in E$ is
\begin{displaymath}
	\left( J^1 E\right)_y = \left\lbrace \gamma \in \text{L}(T_{\pi(y)}M, T_y E) \; \vert \: T\pi \circ \gamma = \text{id}_{T_{\pi(x)} M} \right\rbrace
\end{displaymath}
(T = Tangent functor L = space of linear maps)

which is and affine subspace of $\text{L}(T_{\pi(y)}M, T_y E)$ modelled on the following (difference) vector space:
\begin{displaymath}
	\vec{\left( J^1 E\right)_y} = \left\lbrace z \in \text{L}(T_{\pi(y)}M, T_y E) \; \vert \: T\pi \circ z = 0 \right\rbrace
	\cong text{L}(T_{\pi(y)}M, T_y E)
\end{displaymath}

\section{Spunti / Open problems}
\begin{itemize}
	\item We have a momap on ms mfd and a momap on mapping space (as diffeological space) does the two match?
	\item We have a canonical m-s structure on J.. and a canonical poisson structure con cov phase space (Peierls bracket) does the two match?
	\item there are other notion of moment map on the multisymplectic setting. For instance how the definition in section 4 of \cite{Gotay1998a} is related to the one in \cite{Ryvkin2018}?
\end{itemize}
	


\nocite{*}
\bibliographystyle{alpha}
\bibliography{biblio}



\end{document}
